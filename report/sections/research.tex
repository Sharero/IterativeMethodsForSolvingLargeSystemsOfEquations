\section{Исследование}
Для тестирования будет использоваться набор СЛАУ с комплексными симметричными матрицами, хранящимися в разреженном строчно–столбцовом формате, выданный преподавателем. Набор состоит из 4-х СЛАУ. Для каждой СЛАУ приведены файлы:
\begin{itemize}
	\item \texttt{kuslau.txt} -- текстовый файл, содержащий 3 числа: \texttt{size} -- размер вектора неизвестных, \texttt{epsilon} -- порог уменьшения нормы относительной невязки, \texttt{maximum\_iterations} -- максимальное количество итераций; 
	\item \texttt{b} -- двоичный файл из вещественных чисел типа \texttt{double}, содержащий компоненты вектора правой части; 
	\item \texttt{di} -- двоичный файл из вещественных чисел типа \texttt{double}, содержащий диагональные элементы матрицы; 
	\item \texttt{gg} -- двоичный файл из вещественных чисел типа \texttt{double}, содержащий внедиагональные элементы матрицы; 
	\item \texttt{ig}, \texttt{jg}, \texttt{idi}, \texttt{ijg} -- двоичные файлы из целых чисел типа \texttt{int}, содержащие портрет матрицы; 
\end{itemize}
\par
Полученный набор СЛАУ содержит СЛАУ размерности: 88761, 244425, 1134591, 2963318.
\par
У решателей будет диагональное предобуславливание.
\par
В таблице \ref{tab:1} приведены результаты работы решателей для каждой СЛАУ из набора. Видно, что все методы для первых трех СЛАУ сошлись до требуемой точности $\num{1e-5}$, а для четвертой СЛАУ ни один из методов не смог и завершил работу по достижении максимального количества итераций.
\begin{table}[H]
	\centering
	\caption{-- Результаты работы решателей для всех СЛАУ}
	\label{tab:1}
	\begin{tabular}{|c|c|c|c|c|}
		\hline
		\multirow{2}{*}{Метод} & \multicolumn{4}{c|}{Относительная невязка} \\ \cline{2-5}
		& СЛАУ №1 & СЛАУ №2 & СЛАУ №3 & СЛАУ №4 \\ \hline
		COCG & \num{8.91e-06} & \num{9.76e-06} & \num{6.97e-06} & \num{5.62e-2} \\ \hline
		COCG\_Smooth & \num{9.99e-06} & \num{9.99e-06} & \num{9.86e-06} & \num{1.37e-2} \\ \hline
		COCR & \num{9.84e-06} & \num{9.72e-06} & \num{9.61e-06} & \num{2.96e-2} \\ \hline
		COCR\_Smooth & \num{9.13e-06} & \num{9.95e-06} & \num{9.88e-06} & \num{7.33e-2} \\ \hline
	\end{tabular}
\end{table}
\subsection{Исследование на сходимость методов}
На рисунках \ref{fig:1} - \ref{fig:4} представлены результаты решения метода COCG и COCG со сглаживанием невязки для каждой СЛАУ из набора. Видно, что метод COCG сходится на первых трех СЛАУ, на четвертой СЛАУ не сходится ни COCG, ни COCG со сглаживанием. Также на всех рисунках видно, что COCG без сглаживания невязки сильно осцилирует в процессе решения. Использование алгоритма сглаживания позволило достичь монотонного убывания невязки и небольшого ускорения по итерациям.
\begin{figure}[H]
	\centering
	\includegraphics[width=0.98\textwidth]{./images/residual\_vs\_iterations\_COCG\_1}
	\caption{-- Результаты решения методом COCG для первой СЛАУ}
	\label{fig:1}
\end{figure}
\begin{figure}[H]
	\centering
	\includegraphics[width=0.98\textwidth]{./images/residual\_vs\_iterations\_COCG\_2}
	\caption{-- Результаты решения методом COCG для второй СЛАУ}
	\label{fig:2}
\end{figure}
\begin{figure}[H]
	\centering
	\includegraphics[width=0.98\textwidth]{./images/residual\_vs\_iterations\_COCG\_3}
	\caption{-- Результаты решения методом COCG для третьей СЛАУ}
	\label{fig:3}
\end{figure}
\begin{figure}[H]
	\centering
	\includegraphics[width=0.98\textwidth]{./images/residual\_vs\_iterations\_COCG\_4}
	\caption{-- Результаты решения методом COCG для четвертой СЛАУ}
	\label{fig:4}
\end{figure}
На рисунках \ref{fig:5} - \ref{fig:8} представлены результаты решения метода COCR и COCR со сглаживанием невязки для каждой СЛАУ из набора. Видно, что метод COCR сходится на первых трех СЛАУ, на четвертой СЛАУ не сходится ни COCR, ни COCR со сглаживанием. Также, как и для случая с COCG, использование алгоритма сглаживания позволило достичь монотонного убывания невязки и небольшого ускорения по количеству итераций.
\begin{figure}[H]
	\centering
	\includegraphics[width=0.98\textwidth]{./images/residual\_vs\_iterations\_COCR\_1}
	\caption{-- Результаты решения методом COCR для первой СЛАУ}
	\label{fig:5}
\end{figure}
\begin{figure}[H]
	\centering
	\includegraphics[width=0.98\textwidth]{./images/residual\_vs\_iterations\_COCR\_2}
	\caption{-- Результаты решения методом COCR для второй СЛАУ}
	\label{fig:6}
\end{figure}
\begin{figure}[H]
	\centering
	\includegraphics[width=0.98\textwidth]{./images/residual\_vs\_iterations\_COCR\_3}
	\caption{-- Результаты решения методом COCR для третьей СЛАУ}
	\label{fig:7}
\end{figure}
\begin{figure}[H]
	\centering
	\includegraphics[width=0.98\textwidth]{./images/residual\_vs\_iterations\_COCR\_4}
	\caption{-- Результаты решения методом COCR для четвертой СЛАУ}
	\label{fig:8}
\end{figure}
\subsection{Исследование времени решения СЛАУ в зависимости от  количества потоков}
В таблице \ref{tab:2} представлены результаты тестирования на 1, 2, 4 и 8 вычислительных потоках. Видно, что наилучшее ускорение достигается при 4 потоках. При переходе на 8 потоков, на всех СЛАУ, кроме первой происходит снижение скорости. Разница в ускорении между методами COCG и COCR незначительна.
\begin{landscape}
	\begin{table}[p]
		\centering
		\small
		\caption{-- Результаты работы решателей на всех потоках для всех СЛАУ}
		\label{tab:2}
		\begin{tabular}{|c|cc|cc|cc|cc|}
			\hline
			\multicolumn{9}{|c|}{1 поток} \\ \cline{1-9}
			\multirow{2}{*}{Метод} 
			& \multicolumn{2}{c|}{СЛАУ №1} 
			& \multicolumn{2}{c|}{СЛАУ №2} 
			& \multicolumn{2}{c|}{СЛАУ №3} 
			& \multicolumn{2}{c|}{СЛАУ №4} \\ \cline{2-9}
			& Время & Ускорение & Время & Ускорение & Время & Ускорение & Время & Ускорение \\ \hline
			COCG & 4.65 & 1.0 & 19.79 & 1.0 & 412.71 & 1.0 & 3534.51 & 1.0 \\ \hline
			COCG\_Smooth & 3.70 & 1.0 & 18.62 & 1.0 & 375.41 & 1.0 & 3910.75 & 1.0 \\ \hline
			COCR & 2.85 & 1.0 & 9.15 & 1.0 & 226.17 & 1.0 & 3702.72 & 1.0 \\ \hline
			COCR\_Smooth & 2.43 & 1.0 & 9.77 & 1.0 & 245.40 & 1.0 & 4139.35 & 1.0 \\ \hline
			\multicolumn{9}{|c|}{2 потока} \\ \cline{1-9}
			\multirow{2}{*}{Метод} 
			& \multicolumn{2}{c|}{СЛАУ №1} 
			& \multicolumn{2}{c|}{СЛАУ №2} 
			& \multicolumn{2}{c|}{СЛАУ №3} 
			& \multicolumn{2}{c|}{СЛАУ №4} \\ \cline{2-9}
			& Время & Ускорение & Время & Ускорение & Время & Ускорение & Время & Ускорение \\ \hline
			COCG & 2.67 & 1.74 & 11.51 & 1.72 & 254.94 & 1.62 & 2169.47 &  1.63\\ \hline
			COCG\_Smooth & 2.19 & 1.68 & 11.29 & 1.65 & 237.11 & 1.58 & 2541.19 & 1.54 \\ \hline
			COCR & 1.63 & 1.74 & 5.32 & 1.72 & 139.34 & 1.62 & 2344.80 &  1.58\\ \hline
			COCR\_Smooth & 1.43 & 1.69 & 6.13 & 1.59 & 155.50 & 1.57 & 2670.54 & 1.55 \\ \hline
			\multicolumn{9}{|c|}{4 потока} \\ \cline{1-9}
			\multirow{2}{*}{Метод} 
			& \multicolumn{2}{c|}{СЛАУ №1} 
			& \multicolumn{2}{c|}{СЛАУ №2} 
			& \multicolumn{2}{c|}{СЛАУ №3} 
			& \multicolumn{2}{c|}{СЛАУ №4} \\ \cline{2-9}
			& Время & Ускорение & Время & Ускорение & Время & Ускорение & Время & Ускорение \\ \hline
			COCG & 1.68 & 2.76 & 8.60 & 2.30 & 220.79 & 1.87 & 1857.41 &  1.90\\ \hline
			COCG\_Smooth & 1.39 & 2.66 & 9.44 & 1.97 & 205.55 & 1.83 & 2221.84 & 1.76 \\ \hline
			COCR & 1.05 & 2.71 & 4.07 & 2.25 & 120.06 & 1.88 & 2028.56 &  1.83\\ \hline
			COCR\_Smooth & 0.93 & 2.61 & 5.10 & 1.91 & 134.88 & 1.81 & 2365.34 & 1.75 \\ \hline
			\multicolumn{9}{|c|}{8 потоков} \\ \cline{1-9}
			\multirow{2}{*}{Метод} 
			& \multicolumn{2}{c|}{СЛАУ №1} 
			& \multicolumn{2}{c|}{СЛАУ №2} 
			& \multicolumn{2}{c|}{СЛАУ №3} 
			& \multicolumn{2}{c|}{СЛАУ №4} \\ \cline{2-9}
			& Время & Ускорение & Время & Ускорение & Время & Ускорение & Время & Ускорение \\ \hline
			COCG & 1.32 & 3.52 & 10.50 & 1.88 & 255.28 & 1.62 & 2303.34 &  1.53\\ \hline
			COCG\_Smooth & 1.10 & 3.36 & 10.91 & 1.71 & 235.91 & 1.59 & 2670.89 & 1.46 \\ \hline
			COCR & 0.82 & 3.47 & 4.94 & 1.85 & 141.24 & 1.60 & 2488.01 &  1.49\\ \hline
			COCR\_Smooth & 0.72 & 3.37 &  5.85 & 1.67 & 155.08 & 1.58 & 2815.88 & 1.47 \\ \hline
		\end{tabular}
	\end{table}
\end{landscape}